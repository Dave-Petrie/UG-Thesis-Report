\section{Conclusion}

\paragraph{}
The project being undertaken plans to design and confirm the feasibility of a DC power distribution for commercial buildings to power low load electronics such as lighting and simple devices with an array of photo-voltaic cells. The completion of this task will require extensive research, time, calculations and computer simulations. Milestones that have been set meet the SMART criteria which will allow for tracking and maintaining progress throughout the project. A literature review and analysis of the task has been completed. Additionally, through testing and simulations initial calculations and test modelling have been completed. 

\paragraph{}
Computer simulations are the main design solution due to the large costs involved in commercial power system implementation. By simulating designs and providing visual aids through 3D rendered images, the presentation will be show not only calculation data but designs implemented on a visual model. In the event that an experimental test can be financially and physically completed and it would benefit the task, it will be done.   

\paragraph{}
As of the submission of this progress report (April 2017) calculations have been completed and some conclusions made. The remaining tasks have been outlined in Section \ref{section:remaining-tasks} but are focused around the final stages of the design phase. Thus far, 48\,V\,DC has been deemed the most appropriate voltage level for the system and remaining calculations based off that value. Additionally upon analysis of QUT data and further design and testing initial predictions are that this project should be feasibly. The consumption curves from lighting and production curves from feasible levels of photovoltaic installations appear to be in line such that they could outweigh each other within a DC installation. The created lighting models at this stage also represent consumption suitable for this installation.  

\paragraph{}
Overall it is expected that this research project will be completed with a feasible design. If it is found that no solution will be suitable, a strong justification and possible future areas of discussion will be brought forward. A finalised design will be created in the next stage and final calculations of both efficiencies and finances completed. After these items are completed, all questions will be answered and a final conclusions drawn.  
\newpage