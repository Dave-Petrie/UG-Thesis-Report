%%%%%%%%%%%%%%%
%							METHODS OF PV CALCULATION
%%%%%%%%%%%%%%%

\subsection{Calculations and Modelling of Photovoltaic Systems} \label{section:PV-Calculations}

\paragraph{}
This section is intended to outline the possible calculation methods for appxoimating and modelling expected PV generation. This section will conclude with an approximated production curve to check against the QUT Data and future project model discussed in Section \ref{section:project-model}. There are many ways that industry professionals use to predict and model photovoltaic production from installations however there will always be unknowns and sensitivities causing inaccuracies. Variables such as the weather and luminance can be inconsistent day to day or year to year reducing the accuracy of calculated predictions.

\subsubsection{System Advisor Model}

\paragraph{}
The software package that will be utilised for modelling and testing purposes is known as System Advisor Model (SAM). It is a free software package produced through photovoltaic module and inverter manufacturers funding the programmers. Weather, specific module information, specific inverter information as well as array size and orientations can all be input into the software package and run. This is an incredibly useful tool and will be utilised within the project. Although this is a very useful modelling tool it is unreasonable to rely on one method for caluclations when more are available. 

\subsubsection{Industry Approximation Ratio}

\paragraph{}
From industry experience within Brisbane City, Queensland, there is an estimation ratio of production of 1,500\,kWh/kWp/Annum. What this implies is that for every additional kilowatt peak that the pv array contains, over the year there will be 1,500 additional kilowatt hours of production. This is the first check for designing as the system can be assessed against this value and if it is within plus or minus 20\% sensitivity.  

\subsubsection{Hand Calculation}

\paragraph{}
After simple estimations utilising the industry approximation ratio are completed more accurate hand calculations are also possible. The variables related to this process are related to solar irradiance. The SI units for this value is watts per square meter (W/\si{m^2}). It is known that solar panels will perform at their peak production capability when the irradiance is equal to 1 SUN which is again equal to 1000\,W/\si{m^2}. Irradiance can tell us the number of Peak Sun Hours (PSH) as well. For example, if we know that the irradiance of January is 160,000\,W/\si{m^2} then we know that over one day there is 5,161\,W/\si{m^2} or 5.161\,kW/\si{m^2}. This represents the fact that over the average day in January the illuminance comes to approximately 5.1\,PSH. With these values slightly more accurate approximations can be done specific to geographic locations when irradiance data is made available.       
    