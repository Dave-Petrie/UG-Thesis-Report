\newpage
%\TOCadd{ExecutiveSummary}

\phantomsection

\section*{}

\begin{flushright}
	70 Mary Street\\
	Brisbane, Q 4000\\
	Tel.\ 0400 012 299\\
	\medskip
	\today
\end{flushright}
\begin{flushleft}
  Associate Professor Geoffrey Walker\\
  School of Science and Engineering\\
  Queensland University of Technology\\
  Brisbane City, Q 4000\\
  \bigskip\bigskip
  Dear Professor Walker,
\end{flushleft}

In accordance with the requirements of the degree of Bachelor of
Engineering / Bachelor of Finance (IX28) in the division of Computer Systems
Engineering / Electrical and Electronic Engineering, I present the
following thesis entitled Feasibility Study of Low Voltage Direct Current Power Distribution.  This work was performed under the supervision of Associate Professor Geoffrey Walker.

I declare that the work submitted in this thesis is my own, except as
acknowledged in the text with references, and has not been previously
submitted for a degree at Queensland University of Technology or any other
institution.

\begin{flushright}
	Yours sincerely,\\
	\medskip
	\begin{figure}[H]
	\hfill\includegraphics[width = 40mm]{images/Signature.JPG}%\hspace*{\fill}
	\end{figure} 
	\medskip
	DAVID PETRIE.
\end{flushright}

\newpage

\section*{Acknowledgements}


\paragraph{}
\textbf{Associate Professor Geoffrey Walker} for your guidance, support and patience through the completion of this project. 

\paragraph{}
My fellow students \textbf{Ash Abdullrabzak} and \textbf{Niroj Gurung} for their valuable discussions throughout our three projects. 

\paragraph{}
My family and friends for their support and understanding through the completion of this project and my degree.    

\paragraph{}
Past Queensland University of Technology engineering student \textbf{Steven Donohue} for his valuable prior thesis on 48 V direct current home power systems.  

\paragraph{}
The Electrical Engineering team in Buildings at Aurecon, Brisbane for their technical advice and assistance in the final half of my project.  

\paragraph{}
\textbf{Geoffrey Woods} and \textbf{Norman Higgins} from Queensland University of Technology's Building Management Services for providing technical drawings and electricity data.

\paragraph{}
\textbf{Abhishek Bhasin} and \textbf{Isaac Linett} from Q Electrical for their technical assistance in the first half of my project.



\newpage


\section*{Abstract}

\paragraph{}
This design project aims to explore the issue of incorporating low or extra low voltage direct current (DC) power distribution systems to multi-residential and commercial buildings. Low voltage DC power is prevalent in telecommunications systems but has the possibility of being used for a variety of other applications to reduce electricity costs and improve efficiency. Due to the large costs involved in building power systems, the majority of this design project will be completed through software simulations and hand calculations. If successful, a more efficient and cheaper alternative to running simple electronics from Alternating Current (AC) mains will be created. 

\paragraph{}
To complete this task, the project was separated into individual questions. These questions will cover finer details of the project including the chosen DC voltage level, whether photovoltatics can be used, if lighting requirements will be met, structural design and safety mechanisms for voltage loss reduction and finally whether DC could reasonably replace AC for portions of commercial building power systems. 

\paragraph{}
The QUT data has been analysed and the results are positive. With the compared production and consumption curves that have been determined from the metering sources it appears certainly possible that the photovoltaic production will occur during times where the lighting based consumption occurs. This was expected however it was confirmed via these results. 48\,V\,DC was chosen as the voltage level due to established applications in the real world, simple power calculations as well as commercially available products. Photovoltatics have been determined to be feasible due to the production curves simulated compared against expected lighting consumption curves. The lighting models for the final commercial office floorplan design have been created however full analysis has not yet been completed. From initial calculations of expected lighting requirements, it confirms that at this stage of the project it remains feasible. Final calculations have not been completed at this stage.    
 
\newpage
